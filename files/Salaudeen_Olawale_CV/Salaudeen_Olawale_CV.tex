%%%%%%%%%%%%%%%%%%%%%%%%%%%%%%%%%%%%%%%%%
% Medium Length Professional CV
% LaTeX Template
% Version 2.0 (8/5/13)
%
% This template has been downloaded from:
% http://www.LaTeXTemplates.com
%
% Original author:
% Rishi Shah 
%
% Important note:
% This template requires the resume.cls file to be in the same directory as the
% .tex file. The resume.cls file provides the resume style used for structuring the
% document.
%
%%%%%%%%%%%%%%%%%%%%%%%%%%%%%%%%%%%%%%%%%

%----------------------------------------------------------------------------------------
%	PACKAGES AND OTHER DOCUMENT CONFIGURATIONS

%----------------------------------------------------------------------------------------
\newcommand\tab[1][1cm]{\hspace*{#1}}

\documentclass{resume} % Use the custom resume.cls style
\usepackage{enumitem}
\usepackage{etaremune}

\usepackage{hyperref}
\usepackage[left=0.75in,top=0.6in,right=0.75in,bottom=0.6in]{geometry} % Document margins
\newcommand{\tab}[1]{\hspace{.2667\textwidth}\rlap{#1}}
\newcommand{\itab}[1]{\hspace{0em}\rlap{#1}}
\name{\href{https://olawalesalaudeen.com}{Olawale Salaudeen}} % Your name
% \address{201 N. Goodwin Ave \\ Urbana, IL 61801} % Your address
%\address{123 Pleasant Lane \\ City, State 12345} % Your secondary addess (optional)
\address{\href{https://olawalesalaudeen.com}{https://olawalesalaudeen.com} \\ \href{mailto:olasalaudeen96@gmail.com}{olasalaudeen96@gmail.com} \\ \href{mailto:oes2@illinois.edu}{oes2@illinois.edu}} % Your phone number and email

\usepackage{lastpage}
\usepackage{fancyhdr}
\usepackage{dirtytalk}
\pagestyle{fancy} 
\cfoot{\thepage\ of \pageref{LastPage}}
\renewcommand{\headrulewidth}{0pt}

\begin{document}
\singlespacing

%----------------------------------------------------------------------------------------
%	EDUCATION SECTION
%----------------------------------------------------------------------------------------

\begin{rSection}{Education}

{\bf Stanford University} \hfill {\em September 2022 - Present} 
\\ Visiting Ph.D. Student Researcher
\\ Department of Computer Science
\\Advisor: \href{http://sanmi.cs.illinois.edu/}{Sanmi Koyejo}\\
\\{\bf University of Illinois at Urbana-Champaign} \hfill {\em August 2019 - Present} 
\\ Ph.D. Candidate
\\ Department of Computer Science
\\Advisor: \href{http://sanmi.cs.illinois.edu/}{Sanmi Koyejo}\\
\\{\bf Texas A\&M University} \hfill {\em August 2015 - May 2019} 
\\ Bachelor of Science with Honors, Mechanical Engineering
\\ Minors in Computer Science and Mathematics
\end{rSection}

\begin{rSection}{Research Interests}
Deep Learning, Transfer Learning (Domain Adaptation/Generalization), Causal Inference/Discovery, Causality-Inspired Machine Learning, Probabilistic Graphical Models
\end{rSection}

\begin{rSection}{Publications}
\begin{etaremune}[label={\arabic*.}]

\item \emph{Target Conditioned Representation Independence (TCRI); From Domain-Invariant to Domain-General Representations}\\
\textbf{\underline{Olawale Salaudeen}}, Oluwasanmi Koyejo.\\
In Review.

\item \emph{Adapting to Latent Subgroup Shifts via Concepts and Proxies}\\
Ibrahim Alabdulmohsin, Nicole Chiou, Alexander D'Amour, Arthur Gretton, 
Sanmi Koyejo, Matt J. Kusner, Stephen R. Pfohl, 
\textbf{\underline{Olawale Salaudeen}}, Jessica Schrouff, Katherine Tsai.\\
Authors listed in alphabetical order.\\
In Review.

\item \emph{Addressing Observational Biases in Algorithmic Fairness
Assessments}\\
Chirag Nagpal, \textbf{\underline{Olawale Salaudeen}}, Sanmi Koyejo, Stephen
Pfohl.\\
Conference on Neural Information Processing Systems (NeurIPS), 2022. Workshop
on Algorithmic Fairness through the Lens of Causality and Privacy (AFCP)
(extended abstract)

\item \emph{Adapting to Shifts in Latent Confounders using Observed
Concepts and Proxies}\\
Matt J. Kusner, Ibrahim Alabdulmohsin, Stephen Pfohl, \textbf{\underline{Olawale Salaudeen}},
Arthur Gretton, Sanmi Koyejo, Jessica Schrouff, Alexander D’Amour.\\
International Conference on Machine Learning, 2022. Workshop on Principles of Distribution Shift (PODS)

\item \emph{Ultra-fast 3D fMRI to explore cardiac-induced fluctuations
in BOLD-based functional imaging}\\
Brad Sutton, Aaron Anderson, Benjamin Zimmerman, Paul Camacho, Riwei Jin, Charles Marchini,
\textbf{\underline{Olawale Salaudeen}}, Natalie Ramsy, Davide Boido, Serge
Charpak, Andrew Webb, Luisa Ciobanu.\\
International Society for Magnetic Resonance in Medicine (ISMRM), 2022. (abstract)

\item \emph{Exploiting Causal Chains for Domain Generalization}\\
\textbf{\underline{Olawale Salaudeen}}, Sanmi Koyejo.\\
Conference on Neural Information Processing Systems (NeurIPS), 2021. Workshop on Distribution Shifts -- Connecting Methods and Applications (DistShift)
\end{etaremune}
\end{rSection}

%----------------------------------------------------------------------------------------
%	INTERNSHIPS
%----------------------------------------------------------------------------------------

\begin{rSection}{Professional Experience}
\begin{rSubsection}{Google Research}{May 2022 - Present}{Student Researcher - Cambridge, MA}

\begin{itemize}
    \item Worked on a team to develop a domain adaptation algorithm under latent confounder distribution shift;
    developed semi-synthetic data for evaluation and implemented state-of-the-art domain adaptation algorithms
    \item Developed foundations for generalization bounds under distribution shifts induced 
     by unobserved interventions
    \item Developed a procedure for empirically estimating domain-to-domain transferability from
     unlabeled samples under distribution shifts induced by unobserved interventions
\end{itemize}

\end{rSubsection}

\begin{rSubsection}{Sandia National Laboratories}{May 2017 - April 2022}{Year-Round R\&D Intern - Albuquerque, NM}

\begin{itemize}[label={}]
    \setlength\itemsep{0em}
    \item \textbf{2021.} Developed a deep set predictor with configurable mean and pairwise errors (Type I/II) for multiclass prediction in the context of contraband detection in images
    \item \textbf{2020.} Working on a team to develop models to classify organic materials in X-ray images
    \item \textbf{2020.} Designed and executed experiments to investigate the effectiveness of Reinforcement Learning in sequence to sequence generation -- Deep Q Network in the context of automated code generation
    \item \textbf{2019.} Implemented a rationale generating Recurrent Convolutional Neural Network model for triage classification of triggered network security alerts
    \item \textbf{2019.} Prototyped a Convolutional Neural Network framework for semantic segmentation of X-Ray images of Improvised Explosive Devices and generation of a graphical model of designs of the devices
    \item \textbf{2018.} Developed and implemented a multi-modal deep Recurrent Neural Network framework for classifying safety rules for maintenance tasks from mixed numerical and textual tasks descriptions
    \item \textbf{2018.} Extended a 2D Simultaneous Localization and Mapping (SLAM) algorithm for ground systems to 3D for air systems equipped with 3D-LIDAR, IMU/GPS
    \item \textbf{2017.} Designed and prototyped an intrusion detection and localization system using fiber-optic disturbances 
    \item \textbf{2017.} Researched and presented applications of big data analysis to learn physical properties of a configuration space based on electromagnetic disturbances in transmitted wireless signals
\end{itemize}

\end{rSubsection}
\end{rSection}

%----------------------------------------------------------------------------------------
%	RESEARCH EXPERIENCE SECTION
%----------------------------------------------------------------------------------------

\begin{rSection}{Research Experience}
\begin{rSubsection}{University of Illinois at Urbana-Champaign}{September 2021 - Present}{Miniature Brain Machinery NSF Trainee with Prof. Sanmi Koyejo and Prof. Brad Sutton -- Champaign, Illinois} 

\begin{itemize}
    \setlength\itemsep{0em}
    \item An NSF-funded research traineeship that combines cognitive and behavior studies with brain cell and tissue biology
    \item Developing machine learning algorithms to detect and remove nuisance artifacts, such as the effects of breathing, from fMRI scans
\end{itemize}
\end{rSubsection}
\begin{rSubsection}{University of Illinois at Urbana-Champaign}{August 2020 - July 2021}{Beckman Institute Graduate Research Fellow with Prof. Sanmi Koyejo, Prof. Brad Sutton, and Prof. Aron Barbey -- Champaign, Illinois} 

\begin{itemize}
    \setlength\itemsep{0em}
    \item Developed a causal structure learning framework to isolate and remove motion artifacts in functional Magnetic Resonance Images (fMRI)
\end{itemize}
\end{rSubsection}
\begin{rSubsection}{University of Illinois at Urbana-Champaign}{August 2019 - July 2020}{Graduate Research Assistant with Prof. Sanmi Koyejo and Prof. Aron Barbey -- Champaign, Illinois} 

\begin{itemize}
    \setlength\itemsep{0em}
    \item Developed a learning framework for estimating multi-modal individual treatment effects, correlated changes, and counterfactuals in the context of human performance optimization
\end{itemize}
\end{rSubsection}
\begin{rSubsection}{Texas A\&M University Multi-Robot Laboratory}{October 2018 - May 2019}{Undergraduate Researcher with Prof. Dylan Shell -- College Station, TX}

\begin{itemize}
    \setlength\itemsep{0em}
    \item Created and analyzed a novel geometry-based motion planning algorithm for tethered robots
\end{itemize}
\end{rSubsection}
\begin{rSubsection}{Texas A\&M University Energy Systems Laboratory}{August 2016 - October 2018}{Undergraduate Researcher with Prof. Charles Culp -- College Station, TX}

\begin{itemize}
    \setlength\itemsep{0em}
    \item Developed probabilistic algorithms for fault detection and diagnosis in industrial Heating Ventilation and Air Condition systems
\end{itemize}
\end{rSubsection}
\end{rSection}

%----------------------------------------------------------------------------------------
%	Honors and Awards
%----------------------------------------------------------------------------------------

\begin{rSection}{Honors and Awards}
\begin{itemize}[label={}]
    \setlength\itemsep{0em}
    \item NSF Miniature Brain Machinery Research Trainee \hfill 2021 \\ 
    \textit{University of Illinois at Urbana-Champaign}
    \item GEM Associate Fellow \hfill 2021 \\ 
    \textit{University of Illinois at Urbana-Champaign}
    \item Beckman Institute Graduate Fellow \hfill 2020 \\
    \textit{University of Illinois at Urbana-Champaign}
    \item Sloan Scholar \hfill 2019 \\
    \textit{Alfred P. Sloan Foundation's Minority Ph.D. (MPHD) Program}
    \item Masters Fellowship Program (declined) \hfill 2019 \\
    \textit{Sandia National Laboratories}
    \item Mechanical Engineering Advisory
    Council Scholarship \hfill 2018 \\ \textit{Texas A\&M University}
    \item Foundation Excellence Award \hfill 2017 \\ \textit{Texas A\&M
    University}
    \item Pi Tau Sigma, Sigma Delta \hfill \hfill 2016 \\ \textit{National Mechanical
    Engineering Honors Society} 
    \item Craig and Galen Brown Honors College of Engineering \hfill 2015 \\
    \textit{Texas A\&M University}
    \item Regents Scholar Program \hfill 2015\\ \textit{Texas A\&M University}
\end{itemize}
\end{rSection}

%----------------------------------------------------------------------------------------
%	TALKS SECTION
%----------------------------------------------------------------------------------------

\begin{rSection}{Talks and Presentations}
\begin{etaremune}[label={\arabic*.}]
    \setlength\itemsep{0em}

    \item Denoising via probabilistic graphical model augmentation of ICA-AROMA\\
    \textit{University of Illinois at Urbana-Champaign Beckman Institute Graduate Student Seminar}

    \item Automated Incorporation of Machine Learning (AIM) \\
    \textit{Sandia National Laboratories MARTIANS End of Summer Symposia, 2020}

    \item Interpretable Recurrent Convolutional Neural Networks for Cyber Alert Triaging \\
    \textit{Sandia National Laboratories MARTIANS End of Summer Symposia, 2019}
\end{etaremune}
\end{rSection}

%----------------------------------------------------------------------------------------
%	Service
%----------------------------------------------------------------------------------------

\begin{rSection}{Service}
\textbf{Reviewing}
\begin{itemize}[label={}]
    \setlength\itemsep{0em}
    \item Neural Information Processing Systems (NeurIPS) \hfill 2022
    \item NeurIPS Algorithmic Fairness through the Lens of Causality and Privacy (AFCP) Workshop \hfill 2022
    \item International Conference on Machine Learning (ICML) -- \textit{Top 10\% reviewer award} \hfill 2022
    \item NeurIPS Black In AI (BAI) Workshop \hfill 2021
\end{itemize}

\textbf{University of Illinois at Urbana-Champaign}
\begin{itemize}[label={}]
    \setlength\itemsep{0em}
    \item Directed Reading Program, Mentor \hfill 2022-Present
    \item Graduate Study Committee, 1 of 2 Graduate Student
    Members \hfill 2022
    \item Broadening Participation in Computing, Engagement Subcommittee Member
    \hfill 2021 - 2022
    \item Graduates Engineers Diversifying Illinois, Mentor \hfill
    2020 - 2022
    \item Institute for Inclusion, Diversity, Equity, and
    Access (IDEA), Affiliate Member \hfill 2020 - Present
\end{itemize}

\textbf{Mentorship}
\begin{itemize}[label={}]
    \setlength\itemsep{0em}
    \item Distributed Research Experiences for Undergraduates (DREU) \hfill 2021
\end{itemize}
\end{rSection}

\end{document}
